

%----------------------------------------------------
% SYSU.SPA Beamer Model by DSY May 2024;
% My personal site:  https://destiny200424.github.io/
% or Email me at - dongsy6@mail2.sysu.edu.cn
%----------------------------------------------------


\documentclass[AutoFakeBold,AutoFakeSlant]{beamer}
\usepackage[british]{babel}
\usepackage{graphicx,hyperref,sysu,url}
\usepackage{pstricks-add}
%% 中文
\usepackage{ctex}
\usepackage{newtxmath}
\RequirePackage{booktabs}
\usepackage{multicol} 
\usepackage{multirow}
\usepackage[ruled,linesnumbered]{algorithm2e}
\usepackage{color}

% colors
\definecolor{sysugreen}{RGB}{0,88,38}
\definecolor{sysugreendark}{RGB}{5,71,25}

% commands
\newcommand{\ctoday}{\number\year 年 \number\month 月 \number\day 日}

%% 根据需要修改中文字体
\setCJKmainfont[ItalicFont={SimSun}]{SimSun}
% \setCJKsansfont{Microsoft YaHei}
\setCJKsansfont{SimHei}
\setCJKmonofont{FangSong}

% \setCJKmonofont{SimSun}
% \xeCJKsetcharclass{"0}{"2E7F}{0}
% \xeCJKsetcharclass{"2E80}{"FFFF}{1}

% Require XeLaTeX
\RequirePackage{fontspec,xltxtra,xunicode}
% 根据需要修改英文字体
\setmainfont[Mapping=tex-text]{Times New Roman}
\setsansfont[Mapping=tex-text]{Arial}
\setmonofont{Consolas}

% 设置公式字体
\usefonttheme[onlymath]{serif}


% 演示文稿标题:
%  1 底部显示的标题;
%  2 标题页正中显示的大标题;
\title[这里修改为你的课题(建议与标题一致)]{
  这是大标题} %%%%%%%%%%

% 可选:标题页显示的副标题
\subtitle{副标题} %%%%%%%%%%%

% 作者信息:
%  1 底部显示的作者信息;
%  2 联系信息;
% \author[Author Name]{
%   答辩人 \\\medskip
%   {\small \url{netid@mail2.sysu.edu.cn}} \\
%   {\small \url{www.sysu.edu.cn}}
% }

% 学校学院信息:
%  1 顶部学校信息
%  2 标题页显示的学院学校信息
\institute[Sun Yat-Sen University]{
  中山大学 \\ 物理与天文学院
  }
%%%%%%%%%%%%%%%
%% 时间信息
%  1 底部显示的标准时间信息
%  2 标题页显示的中文时间信息
\date[\today]{\ctoday}

\begin{document}

% 作者信息%%%%%%%%%%%%%%%
\author[netid@mail2.sysu.edu.cn]{
  主讲人  \ XXXX
  \texorpdfstring{\\\medskip {\small 导师 \quad XXX 教授 }}{XXX}
}

\begin{frame}
  % 标题页
  \titlepage
\end{frame}
% 标题页不编号
\setcounter{framenumber}{0}

\begin{frame}
  % 目录页
  \frametitle{目\quad 录}
  % 双栏
  % \begin{multicols}{2}
  %   \tableofcontents
  % \end{multicols}
  \tableofcontents[hideallsubsections]
\end{frame}

\section{引言}

\begin{frame}
  \frametitle{介绍}

\end{frame}

\begin{frame}
  \frametitle{内容}

  引言内容包括:
  \begin{enumerate}
    \item \textbf{本研究课题的学术背景及理论与实际意义};
    \item \textbf{本研究课题的来源及主要研究内容};
    \item \textbf{建立研究的线索与思路}。
  \end{enumerate}

\end{frame}

\section{背景}

\subsection{XXX}

\begin{frame}
  \frametitle{XXXX}

\end{frame}

\begin{frame}
  \frametitle{Why Beamer}
	\begin{itemize}
		\item \LaTeX 广泛用于学术界,期刊会议论文模板
	\end{itemize}
	\begin{table}[h]
		\centering
		\begin{tabular}{c|c}
			Microsoft\textsuperscript{\textregistered}  Word & \LaTeX \\
			\hline
			文字处理工具 & 专业排版软件 \\
			容易上手,简单直观 & 容易上手 \\
			所见即所得 & 所见即所想,所想即所得 \\
			高级功能不易掌握 & 进阶难,但一般用不到 \\
			处理长文档需要丰富经验 & 和短文档处理基本无异 \\
			花费大量时间调格式 & 无需担心格式,专心作者内容 \\
			公式排版差强人意 & 尤其擅长公式排版 \\
			二进制格式,兼容性差 & 文本文件,易读、稳定 \\
			付费商业许可 & 自由免费使用 \\
		\end{tabular}
	\end{table}
\end{frame}


\section{正文}

\subsection{步骤一}

\begin{frame}
  \frametitle{Say my name}
 \begin{figure}[H]
 	\centering
 	\includegraphics[width=0.7\linewidth]{figures/say_my_name}
 	\label{fig:saymyname}
 \end{figure}
 
\end{frame}

\subsection{步骤二}

\begin{frame}
  \frametitle{添加图片}
\begin{figure}[H]
	\centering
	\includegraphics[width=0.5\linewidth]{figures/gzh}
	\caption{欢迎关注哈哈}
	\label{fig:gzh}
\end{figure}
\end{frame}

\section{举例}

\subsection{XXXX}

\begin{frame}
  \frametitle{排版举例}
  \begin{exampleblock}{无编号公式} % 加 * 
  	\begin{equation*}
  		J(\theta) = \mathbb{E}_{\pi_\theta}[G_t] = \sum_{s\in\mathcal{S}} d^\pi (s)V^\pi(s)=\sum_{s\in\mathcal{S}} d^\pi(s)\sum_{a\in\mathcal{A}}\pi_\theta(a|s)Q^\pi(s,a)
  	\end{equation*}
  \end{exampleblock}
  \begin{exampleblock}{多行多列公式\footnote{\color{white}如果公式中有文字出现,请用 $\backslash$mathrm\{\} 或者 $\backslash$text\{\} 包含,不然就会变成 $clip$,在公式里看起来比 $\mathrm{clip}$ 丑非常多。}}
  	% 使用 & 分隔
  	\begin{align}
  		Q_\mathrm{target}&=r+\gamma Q^\pi(s^\prime, \pi_\theta(s^\prime)+\epsilon)\\
  		\epsilon&\sim\mathrm{clip}(\mathcal{N}(0, \sigma), -c, c)\nonumber
  	\end{align}
  \end{exampleblock}
\end{frame}

\begin{frame}
	\frametitle{图形与分栏}
	% From thuthesis user guide.
	\begin{minipage}[c]{0.3\linewidth}
		\psset{unit=0.8cm}
		\begin{pspicture}(-1.75,-3)(3.25,4)
			\psline[linewidth=0.25pt](0,0)(0,4)
			\rput[tl]{0}(0.2,2){$\vec e_z$}
			\rput[tr]{0}(-0.9,1.4){$\vec e$}
			\rput[tl]{0}(2.8,-1.1){$\vec C_{ptm{ext}}$}
			\rput[br]{0}(-0.3,2.1){$\theta$}
			\rput{25}(0,0){%
				\psframe[fillstyle=solid,fillcolor=lightgray,linewidth=.8pt](-0.1,-3.2)(0.1,0)}
			\rput{25}(0,0){%
				\psellipse[fillstyle=solid,fillcolor=yellow,linewidth=3pt](0,0)(1.5,0.5)}
			\rput{25}(0,0){%
				\psframe[fillstyle=solid,fillcolor=lightgray,linewidth=.8pt](-0.1,0)(0.1,3.2)}
			\rput{25}(0,0){\psline[linecolor=red,linewidth=1.5pt]{->}(0,0)(0.,2)}
			%           \psRotation{0}(0,3.5){$\dot\phi$}
			%           \psRotation{25}(-1.2,2.6){$\dot\psi$}
			\psline[linecolor=red,linewidth=1.25pt]{->}(0,0)(0,2)
			\psline[linecolor=red,linewidth=1.25pt]{->}(0,0)(3,-1)
			\psline[linecolor=red,linewidth=1.25pt]{->}(0,0)(2.85,-0.95)
			\psarc{->}{2.1}{90}{112.5}
			\rput[bl](.1,.01){C}
		\end{pspicture}
	\end{minipage}\hspace{1cm}
	\begin{minipage}{0.3\linewidth}
		\medskip
		%\hspace{2cm}
		\begin{figure}[h]
			\centering
			\includegraphics[height=.3\textheight]{logo.png}
		\end{figure}
	\end{minipage}
\end{frame}

\begin{frame}
	    \begin{exampleblock}{编号多行公式}
		% Taken from Mathmode.tex
		\begin{multline}
			A=\lim_{n\rightarrow\infty}\Delta x\left(a^{2}+\left(a^{2}+2a\Delta x+\left(\Delta x\right)^{2}\right)\right.\label{eq:reset}\\
			+\left(a^{2}+2\cdot2a\Delta x+2^{2}\left(\Delta x\right)^{2}\right)\\
			+\left(a^{2}+2\cdot3a\Delta x+3^{2}\left(\Delta x\right)^{2}\right)\\
			+\ldots\\
			\left.+\left(a^{2}+2\cdot(n-1)a\Delta x+(n-1)^{2}\left(\Delta x\right)^{2}\right)\right)\\
			=\frac{1}{3}\left(b^{3}-a^{3}\right)
		\end{multline}
	\end{exampleblock}
\end{frame}




\section{XXX}

\subsection{工作总结}
\begin{frame}
  \frametitle{\LaTeX{} 常用命令}
  \begin{exampleblock}{命令}
  	\centering
  	\footnotesize
  	\begin{tabular}{llll}
  		\cmd{chapter} & \cmd{section} & \cmd{subsection} & \cmd{paragraph} \\
  		章 & 节 & 小节 & 带题头段落 \\\hline
  		\cmd{centering} & \cmd{emph} & \cmd{verb} & \cmd{url} \\
  		居中对齐 & 强调 & 原样输出 & 超链接 \\\hline
  		\cmd{footnote} & \cmd{item} & \cmd{caption} & \cmd{includegraphics} \\
  		脚注 & 列表条目 & 标题 & 插入图片 \\\hline
  		\cmd{label} & \cmd{cite} & \cmd{ref} \\
  		标号 & 引用参考文献 & 引用图表公式等\\\hline
  	\end{tabular}
  \end{exampleblock}
  \begin{exampleblock}{环境}
	\centering
	\footnotesize
	\begin{tabular}{lll}
		\env{table} & \env{figure} & \env{equation}\\
		表格 & 图片 & 公式 \\\hline
		\env{itemize} & \env{enumerate} & \env{description}\\
		无编号列表 & 编号列表 & 描述 \\\hline
	\end{tabular}
\end{exampleblock}
\end{frame}

\subsection{研究展望}
\begin{frame}
  \frametitle{研究展望}

  \begin{enumerate}
    \item 针对问题XXX。
  \end{enumerate}
\end{frame}

\section{参考文献}

\begin{frame}
	  \frametitle{Reference}
%	\bibliography{ref}
%	\bibliographystyle{alpha}
	% 如果参考文献太多的话,可以像下面这样调整字体:
	% \tiny\bibliographystyle{alpha}
	
	%%%% 注意需要自己去弄lib
\end{frame}

\begin{frame}[plain]
  
    \LARGE{
   谢谢!}
  
\end{frame}

\end{document}
