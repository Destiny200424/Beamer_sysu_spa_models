

%----------------------------------------------------
% SYSU.SPA Beamer Model by DSY May 2024;
% My personal site:  https://destiny200424.github.io/
% or Email me at - dongsy6@mail2.sysu.edu.cn
%----------------------------------------------------


\documentclass[AutoFakeBold,AutoFakeSlant]{beamer}
\usepackage[british]{babel}
\usepackage{graphicx,hyperref,sysu,url}

%% 中文
\usepackage{ctex}
\usepackage{newtxmath}
\RequirePackage{booktabs}
\usepackage{multicol} 
\usepackage{multirow}
\usepackage[ruled,linesnumbered]{algorithm2e}
\usepackage{color}

% colors
\definecolor{sysugreen}{RGB}{0,88,38}
\definecolor{sysugreendark}{RGB}{5,71,25}

% commands
\newcommand{\ctoday}{\number\year 年 \number\month 月 \number\day 日}

%% 根据需要修改中文字体
\setCJKmainfont[ItalicFont={SimSun}]{SimSun}
% \setCJKsansfont{Microsoft YaHei}
\setCJKsansfont{SimHei}
\setCJKmonofont{FangSong}

% \setCJKmonofont{SimSun}
% \xeCJKsetcharclass{"0}{"2E7F}{0}
% \xeCJKsetcharclass{"2E80}{"FFFF}{1}

% Require XeLaTeX
\RequirePackage{fontspec,xltxtra,xunicode}
% 根据需要修改英文字体
\setmainfont[Mapping=tex-text]{Times New Roman}
\setsansfont[Mapping=tex-text]{Arial}
\setmonofont{Consolas}

% 设置公式字体
\usefonttheme[onlymath]{serif}


% 演示文稿标题:
%  1 底部显示的标题;
%  2 标题页正中显示的大标题;
\title[这里修改为你的课题(建议与标题一致)]{
  这是大标题} %%%%%%%%%%

% 可选:标题页显示的副标题
\subtitle{副标题} %%%%%%%%%%%

% 作者信息:
%  1 底部显示的作者信息;
%  2 联系信息;
% \author[Author Name]{
%   答辩人 \\\medskip
%   {\small \url{netid@mail2.sysu.edu.cn}} \\
%   {\small \url{www.sysu.edu.cn}}
% }

% 学校学院信息:
%  1 顶部学校信息
%  2 标题页显示的学院学校信息
\institute[Sun Yat-Sen University]{
  中山大学 \\ 物理与天文学院
  }
%%%%%%%%%%%%%%%
%% 时间信息
%  1 底部显示的标准时间信息
%  2 标题页显示的中文时间信息
\date[\today]{\ctoday}

\begin{document}

% 作者信息%%%%%%%%%%%%%%%
\author[netid@mail2.sysu.edu.cn]{
  主讲人  \ XXXX
  \texorpdfstring{\\\medskip {\small 导师 \quad XXX 教授 }}{XXX}
}

\begin{frame}
  % 标题页
  \titlepage
\end{frame}
% 标题页不编号
\setcounter{framenumber}{0}

\begin{frame}
  % 目录页
  \frametitle{目\quad 录}
  % 双栏
  % \begin{multicols}{2}
  %   \tableofcontents
  % \end{multicols}
  \tableofcontents[hideallsubsections]
\end{frame}

\section{引言}

\begin{frame}
  \frametitle{介绍}

\end{frame}

\begin{frame}
  \frametitle{内容}

  引言内容包括:
  \begin{enumerate}
    \item \textbf{本研究课题的学术背景及理论与实际意义};
    \item \textbf{本研究课题的来源及主要研究内容};
    \item \textbf{建立研究的线索与思路}。
  \end{enumerate}

\end{frame}

\section{背景}

\subsection{XXX}

\begin{frame}
  \frametitle{列表示例}
\begin{itemize}
	\item \LaTeX 广泛用于学术界,期刊会议论文模板
\end{itemize}
\begin{table}[h]
	\centering
	\begin{tabular}{c|c}
		Microsoft\textsuperscript{\textregistered}  Word & \LaTeX \\
		\hline
		文字处理工具 & 专业排版软件 \\
		容易上手,简单直观 & 容易上手 \\
		所见即所得 & 所见即所想,所想即所得 \\
		高级功能不易掌握 & 进阶难,但一般用不到 \\
		处理长文档需要丰富经验 & 和短文档处理基本无异 \\
		花费大量时间调格式 & 无需担心格式,专心作者内容 \\
		公式排版差强人意 & 尤其擅长公式排版 \\
		二进制格式,兼容性差 & 文本文件,易读、稳定 \\
		付费商业许可 & 自由免费使用 \\
	\end{tabular}
\end{table}
\end{frame}

\begin{frame}
  \frametitle{一个简单的色块}
  就可以根据这个改
 \begin{exampleblock}{无编号公式} % 加 * 
	\begin{equation*}
		J(\theta) = \mathbb{E}_{\pi_\theta}[G_t] = \sum_{s\in\mathcal{S}} d^\pi (s)V^\pi(s)=\sum_{s\in\mathcal{S}} d^\pi(s)\sum_{a\in\mathcal{A}}\pi_\theta(a|s)Q^\pi(s,a)
	\end{equation*}
\end{exampleblock}
\end{frame}


\section{正文}

\subsection{步骤一}

\begin{frame}
  \frametitle{图片插入}
 
 \begin{figure}[H]
 	\centering
 	\includegraphics[width=0.5\linewidth]{figures/gzh}
 	\caption{欢迎关注哈哈}
 	\label{fig:gzh}
 \end{figure}
 
\end{frame}

\subsection{步骤二}

\begin{frame}
  \frametitle{XXX}

\end{frame}

\section{总结展望}

\subsection{工作总结}
\begin{frame}
  \frametitle{工作总结}

  \begin{enumerate}
    \item 本文提出XXX。
  \end{enumerate}
\end{frame}

\subsection{研究展望}
\begin{frame}
  \frametitle{研究展望}

  \begin{enumerate}
    \item 针对问题XXX。
  \end{enumerate}
\end{frame}

\section{参考文献}
\begin{frame}
	\frametitle{参考文献}
	%	\bibliography{ref}
	%	\bibliographystyle{alpha}
	% 如果参考文献太多的话,可以像下面这样调整字体:
	% \tiny\bibliographystyle{alpha}
	
	%%%% 注意需要自己去弄lib
\end{frame}



\begin{frame}[plain]
  
    \LARGE{
   谢谢!}
  
\end{frame}

\end{document}
